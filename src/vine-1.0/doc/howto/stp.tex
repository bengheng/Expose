\subsection {Querying STP}

Now, in the last step we wish to ask the question ``what input values
force the execution down the path taken in the execution?''.  In the
formula we've built, this is equivalent to asking for a set of
assignments that make the variable \verb'post' true. We use STP to
solve this formula for us.  The STP file has the symbolic
\verb'INPUT' variable marked free (along with the initial contents of
memory), and it asserts that the final value of {\tt post} is true.

A symbolic formula $F$ is \emph{valid} if it is true in all
interpretations.  In other words, $F$ is valid if all assignments to
the free (symbolic) variables make $F$ true. Given a formula, STP
decides whether it is valid or not. If it is invalid, then there
exists at least one set of inputs that make the formula false, and
STP can report such an assignment (a {\em counterexample}). We use
this feature to get the assignment to the free \verb'INPUT' variable
in the formula that makes the execution follow the traced path.
Since we don't need to impose any additional constraints, beyond the
ones included in {\tt post}, the formula we ask STP to try to falsify
is {\tt FALSE}, which should be easily to falsify as long as the
constraints are satisfiable.

To do this, we add the following 2 lines at the end of the STP file
and run STP on it:

\begin{Verbatim}[frame=lines, framesep=.5em]
% cat >>five.stp
QUERY(FALSE);
COUNTEREXAMPLE;
% ./stp/stp five.stp
Invalid.
ASSERT( INPUT_1001_0_61  = 0hex35  );
\end{Verbatim}

STP's reply of \verb'Invalid.' indicates it has determined that the
query formula \verb'FALSE' is not valid: there is an assignment to the
program inputs that satisfies the other assertions in the file (i.e.,
would lead the program to execute the same path that was observed),
but still leaves \verb'FALSE' false. As a counterexample it gives one
such input (in this case, the only possible one), in which the input
has the hex value \verb'0x35' (ASCII for \texttt{5}).


