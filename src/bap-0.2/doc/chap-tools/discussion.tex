\section{Discussion}
\label{vine:discussion}

\subsection{Why Design a New Infrastructure?}  

At a high level, we designed \bap as a new platform because existing
platforms are a) defined for higher-level languages and thus not
appropriate for binary code analysis, b) designed for orthogonal goals
such as binary instrumentation or decompilation, and/or c) unavailable
to use for research purposes.  As a result, other tools we tried
(e.g., DynInst~\cite{dyninst} version 5.2, Phoenix~\cite{phoenix}
April 2008 SDK, IDA Pro~\cite{idapro} version 5, and others) were
inadequate, e.g., could not create a correct control flow graph for
the 3 line assembly shown in Figure~\ref{vine:shl-add}. Another reason
we created \bap is so that we could be sure of the semantics of 
analysis. Existing platforms tend not to have publicly available
formally defined semantics, thus we would not know exactly what we are
using.

Formal semantics are important for several practical reasons. We found
that defining the semantics of \bap was helpful in catching design
bugs, unsupported assumptions, and other errors that could affect many
different kinds of analysis. In addition, the semantics are helpful
for communicating how \bap works with other researchers.  Further,
without a specified semantics, it is difficult to show an analysis is
correct. For example, in ~\cite{brumley:2006:alias} we show our
proposed assembly-level alias analysis~\cite{brumley:2006:alias} is
correct with respect to the operational semantics of \bap.


\subsection{Limitations of \bap} 

\bap is designed to enable program analysis of binary program
states. Therefore, analyses that depends upon more than the
operational semantics of the instruction fall outside the scope of
\bap. For example, creating an analysis of the timing behavior of
binary programs falls outside the current scope of \bap.

Some architectures, such as later versions of ARM, are bi-endian where
the endianness is specified by a register status flag and can be
changed mid-program. \bap currently supports big and little endian,
but does not support bi-endian architectures. Adding support is
straight-forward by changing {\tt store} and {\tt load} instructions
to have an additional field indicating the endianness (which can then
be removed by normalizing memory appropriately).  If we annotate
individual loads and stores, we would need to remove the endianness as
part of the memory type declaration in order to prevent the semantics
allowing type-incompatible loads and stores on memory.

\bap does not strongly associated an assembly instruction to the
corresponding sequence of lifted IL instructions.  Currently \bap
prefixes a sequence of IL instructions with a comment containing the
pretty-printed assembly.  However, an analysis itself does not know
which IL instructions arose from which assembly.  Although users of
\bap often seem to want to tie IL instructions to assembly
instructions, it is unclear how to accomplish this in flexible way
that is also non-intrusive to analyses themselves.

One principle in \bap is to  note explicitly in the IL any unhandled
instructions during lifting.  For each architecture supported, there
may be many instructions which \bap cannot lift.  Such instructions
are noted in the IL as either {\tt special} or {\tt unknown}. For
example, \bap does not currently support floating point numbers. The
central reason lies in properly defining the semantics of floating
point with respect to rounding.  This is a problem we plan on tackling
in future versions of \bap.

\subsection{Is Lifting Correct?}

Assembly instructions are lifted to \bap in a syntax directed manner.
One may view that the \bap IL is a model of the underlying assembly.
There is a chance, however, that lifting could produce incorrect
IL. Although it is impossible to say that all assembly instructions
are correctly lifted, the advantage of \bap's design is that only the
lifting process needs to understand the semantics of the original
assembly.


We have developed several measures to make sure our IL is
correct. First, we have a series of unit tests for IL translations. A
unit test executes both the real instruction and the lifted IL in a
know pre-state and verifies both terminate in the same post-state.
Second, we have performed larger scale tests that verify the
executions consisting of millions of instructions produce the same
output value on the corresponding IL.  


\subsection{Size of Lifting IL for a Program}

A single assembly instruction will typically be translated into
several \bap instructions. Thus, the resulting \bap program will
have more statements than the corresponding assembly.  For example and
roughly speaking, x86 assembly instructions are raised to be about 7
\bap instructions: 1 \bap instruction for the direct effect, and 6
for updating processor-specific status flags.

In our experience, the constant-size factor in code size is worth the
benefits of simplifying the semantics of assembly.  Assembly
instructions are designed to be efficient for a computer to execute,
not for a human to understand. The IL, on the other hand, is designed
to be easy for a human to understand.  We have found even experienced
assembly-level programmers will comment that they have a hard time
keeping track of control and data dependencies since there are few
syntactic cues in assembly to help. \bap, on the other hand, obviates
all data and control dependencies within the code. 




